\chapter{Polarized Protocols: The Primitive Structure}\label{chap:polarized_protocols}

\begin{chapterintro}
This chapter establishes protocols as the primitive mathematical notion from which all other structures emerge. Rather than treating protocols as "models of" or "representations of" classical objects, we develop protocol theory \emph{ab initio}, showing that concepts like sets, functions, topological spaces, and even logic itself are \emph{derived} constructions within protocol theory.

The key innovation is \emph{polarization}---an asymmetry between positive (data-producing) and negative (observation-demanding) positions that has no analogue in classical foundations but pervades computation, interaction, and physical measurement. This asymmetry is not a feature we add to mathematics; it is the fundamental structure from which mathematics arises.

We prove that protocols form a $*$-autonomous category with trace structure, establish the Polarization Decomposition Theorem showing every protocol canonically factors into positive, negative, and neutral components, and demonstrate that classical mathematics embeds via a "depolarization" functor that forgets this crucial structure.
\end{chapterintro}

\section{Primitive Notions: Positions, Polarity, and Plays}

\subsection{The Fundamental Asymmetry}

Classical foundations treat mathematical objects as static entities: sets contain elements, functions map inputs to outputs, topological spaces have open sets. But this misses something essential: \emph{interaction has direction}. When you query a data structure, you demand information; when it responds, it produces information. This asymmetry---\emph{demand} versus \emph{production}---cannot be captured by symmetric, static structures like sets.

\begin{definition}[Polarity]\label{def:ch1:polarity}
A \emph{polarity} is a two-element set $\{+, -\}$ equipped with an involution $\overline{(-)}: \{+,-\} \to \{+,-\}$ where $\overline{+} = -$ and $\overline{-} = +$.

We call $+$ the \emph{positive polarity} (production, output, data provision) and $-$ the \emph{negative polarity} (demand, input, observation).
\end{definition}

\begin{remark}
Polarity is not a mere labeling convention---it encodes the fundamental asymmetry of interaction. In computation: functions \emph{demand} inputs ($-$) and \emph{produce} outputs ($+$). In physics: measurements \emph{demand} observables ($-$) and systems \emph{produce} values ($+$). In logic: assumptions are negative (demanded premises), conclusions are positive (produced claims).
\end{remark}

\begin{definition}[Protocol Arena]\label{def:ch1:arena}
A \emph{protocol arena} $A$ consists of:
\begin{enumerate}[(i)]
\item A countable set $|A|$ of \emph{positions} (also called moves or events)
\item A partial order $\leq_A$ on $|A|$ (the \emph{enabling relation}): $m \leq_A n$ means "$m$ must occur before $n$ can occur"
\item A \emph{polarity labeling} $\lambda_A: |A| \to \{+, -\}$ assigning each position a polarity
\item A distinguished \emph{initial position} $\iota_A \in |A|$ with $\lambda_A(\iota_A) = -$
\end{enumerate}
satisfying:
\begin{enumerate}[(P1)]
\item \textbf{(Alternation)} If $m <_A n$ (strict order), then $\lambda_A(m) \neq \lambda_A(n)$
\item \textbf{(Well-foundedness)} $\leq_A$ is well-founded: no infinite descending chains
\item \textbf{(Finite branching)} For each $m \in |A|$, the set $\{n : m <_A n\}$ is finite
\item \textbf{(Rootedness)} For all $m \in |A|$, there exists a chain $\iota_A = m_0 <_A m_1 <_A \cdots <_A m_k = m$
\end{enumerate}
\end{definition}

\begin{remark}[Why These Axioms?]
\begin{itemize}
\item \textbf{Alternation (P1)} enforces genuine dialogue: no role can make arbitrarily many consecutive moves. This distinguishes interaction from monologue.
\item \textbf{Well-foundedness (P2)} ensures every interaction eventually progresses; no infinite backtracking.
\item \textbf{Finite branching (P3)} makes protocols computationally tractable; infinite branching would require uncountable oracles.
\item \textbf{Rootedness (P4)} ensures every position is reachable from the initial demand.
\end{itemize}
These are not arbitrary restrictions---they are the minimal structure needed to capture \emph{effective interaction}.
\end{remark}

\begin{example}[Fundamental Arenas]\label{ex:ch1:basic_arenas}
\begin{enumerate}[(a)]
\item \textbf{The unit arena $\mathbf{1}$}: $|\mathbf{1}| = \{\iota\}$ with $\lambda(\iota) = -$. This is the "trivial demand" protocol.

\item \textbf{The empty arena $\mathbf{0}$}: $|\mathbf{0}| = \emptyset$. This is the "impossible protocol" (no positions exist, not even an initial demand).

\item \textbf{Natural number sampling $\mathbb{N}$}: 
\[
|\mathbb{N}| = \{\iota\} \cup \{n : n \in \omega\}, \quad \lambda(\iota) = -, \, \lambda(n) = +, \quad \iota <_{\mathbb{N}} n \text{ for all } n
\]
The initial demand $\iota$ enables any natural number response. This is the protocol-theoretic definition of $\mathbb{N}$.

\item \textbf{Function space $A \Rightarrow B$}:
\[
|A \Rightarrow B| = \{\ iota\} \cup |A|^+ \cup |B|^-
\]
where $|A|^+$ are $A$'s positions with polarity flipped to $+$ (arguments are provided) and $|B|^-$ are $B$'s positions with native polarity $-$ (results are demanded). Order: $\iota < a$ for all $a \in |A|^+$, and $a < b$ for $a \in |A|^+, b \in |B|^-$.

This is the protocol-theoretic definition of function space---note the asymmetry: arguments are positive (given), results are negative (demanded).
\end{enumerate}
\end{example}

\subsection{Plays and Strategies}

\begin{definition}[Play]\label{def:ch1:play}
A \emph{play} in arena $A$ is a finite sequence $s = m_0 m_1 \cdots m_k$ where:
\begin{enumerate}[(i)]
\item $m_0 = \iota_A$ (starts at initial position)
\item $m_i <_A m_{i+1}$ for all $i < k$ (respects enabling)
\item Polarities alternate: $\lambda_A(m_i) \neq \lambda_A(m_{i+1})$ (by alternation axiom)
\end{enumerate}
We write $\mathcal{P}(A) \subseteq |A|^*$ for the set of all plays in $A$.
\end{definition}

\begin{remark}
Plays are the \emph{possible interaction sequences}. A play is not yet a protocol---it's a single possible execution trace.
\end{remark}

\begin{definition}[Strategy]\label{def:ch1:strategy}
A \emph{strategy} $\sigma$ on arena $A$ is a prefix-closed set $\sigma \subseteq \mathcal{P}(A)$ satisfying:
\begin{enumerate}[(S1)]
\item \textbf{(Non-empty)} $\iota_A \in \sigma$
\item \textbf{(Prefix-closed)} If $s \cdot m \in \sigma$ then $s \in \sigma$
\item \textbf{(Determinacy)} If $s \in \sigma$ ends with negative polarity, then there is at most one $m$ with $s \cdot m \in \sigma$
\end{enumerate}
\end{definition}

\begin{remark}[Asymmetric Determinacy]
Condition (S3) is crucial: strategies respond deterministically to negative positions (demands) but the environment can provide any valid positive response. This asymmetry is fundamental---strategies are not arbitrary sets of plays but \emph{algorithmic response patterns}.
\end{remark}

\begin{definition}[Protocol]\label{def:ch1:protocol}
A \emph{protocol} is a pair $P = (A_P, \Sigma_P)$ where $A_P$ is an arena and $\Sigma_P$ is a strategy on $A_P$.

We often write just $P$ when the arena and strategy are clear from context, writing $|P|$ for $|A_P|$ and $\lambda_P$ for $\lambda_{A_P}$.
\end{definition}

\begin{theorem}[Protocols Form a Category]\label{thm:ch1:prot_category}
There exists a category $\Prot$ where:
\begin{itemize}
\item \textbf{Objects} are protocols $P = (A_P, \Sigma_P)$
\item \textbf{Morphisms} $\sigma: P \to Q$ are protocol refinements (defined below)
\item \textbf{Identities} $\id_P$ are copycat strategies
\item \textbf{Composition} is given by parallel composition plus hiding
\end{itemize}
satisfying all category axioms.
\end{theorem}

% PROOF_TODO: thm:ch1:prot_category

\subsection{Duality: The Fundamental Involution}

\begin{definition}[Arena Duality]\label{def:ch1:arena_dual}
For arena $A$, define the \emph{dual arena} $A^{\perp}$ by:
\begin{itemize}
\item $|A^{\perp}| = |A|$ (same positions)
\item $\leq_{A^{\perp}} = \leq_A$ (same enabling order)
\item $\lambda_{A^{\perp}}(m) = \overline{\lambda_A(m)}$ (flip all polarities)
\item $\iota_{A^{\perp}} = \iota_A$ (same initial position, now positive!)
\end{itemize}
\end{definition}

\begin{theorem}[Involutive Duality]\label{thm:ch1:involutive_duality}
For any arena $A$:
\begin{enumerate}[(a)]
\item $(A^{\perp})^{\perp} = A$ (duality is involutive)
\item $|A^{\perp}| = |A|$ (same cardinality)
\item $\dim(A^{\perp}) = \dim(A)$ where $\dim$ is protocol dimension (defined later)
\end{enumerate}
\end{theorem}

% PROOF_TODO: thm:ch1:involutive_duality

\begin{remark}[Duality is Not Complement]
In set theory, complement $X^c$ satisfies $X \cap X^c = \emptyset$. But in protocol theory, $A$ and $A^{\perp}$ have the \emph{same} positions---only polarities flip. This is interaction-theoretic duality: what was demanded becomes produced, what was produced becomes demanded.

This is the protocol-theoretic analogue of De Morgan duality in logic, but for interaction rather than propositions.
\end{remark}

\begin{definition}[Strategy Duality]\label{def:ch1:strategy_dual}
For strategy $\sigma$ on arena $A$, the \emph{dual strategy} $\sigma^{\perp}$ on $A^{\perp}$ is defined by:
\[
s \in \sigma^{\perp} \iff s \in \sigma \text{ (same plays, interpreted in dual arena)}
\]
Since polarities are flipped, what was a negative position (strategy moves) in $A$ becomes positive (environment moves) in $A^{\perp}$, and vice versa.
\end{definition}

\begin{theorem}[Strategy-Environment Duality]\label{thm:ch1:strategy_env_dual}
There is a natural bijection:
\[
\{\text{Strategies on } A\} \leftrightarrow \{\text{Co-strategies on } A^{\perp}\}
\]
given by $\sigma \mapsto \sigma^{\perp}$, where co-strategies respond to positive positions (environment role) rather than negative positions (strategy role).

Moreover, composition satisfies:
\[
(\sigma \otimes \tau)^{\perp} = \sigma^{\perp} \Par \tau^{\perp}
\]
where $\otimes$ is tensor (simultaneous provision) and $\Par$ is its dual (simultaneous demand).
\end{theorem}

% PROOF_TODO: thm:ch1:strategy_env_dual

\begin{corollary}[Interaction Symmetry at Meta-Level]
Although individual protocols are polarized (asymmetric), the theory of protocols is symmetric: every protocol has a dual representing its environment. Composition of protocol with its dual environment yields interaction.
\end{corollary}

\section{The Tensor Product and $*$-Autonomous Structure}

\subsection{Parallel Composition: Tensor Product}

\begin{definition}[Protocol Tensor Product]\label{def:ch1:tensor}
For arenas $A$ and $B$, define $A \otimes B$ by:
\begin{itemize}
\item $|A \otimes B| = |A| \sqcup |B|$ (disjoint union of positions)
\item $\leq_{A \otimes B}$ is the disjoint union of $\leq_A$ and $\leq_B$ (no cross-dependencies)
\item $\lambda_{A \otimes B}$ agrees with $\lambda_A$ on $|A|$ and $\lambda_B$ on $|B|$
\item $\iota_{A \otimes B}$ is a new negative position enabling both $\iota_A$ and $\iota_B$
\end{itemize}

For strategies, define $\sigma \otimes \tau$ as the set of plays interleaving plays from $\sigma$ and $\tau$ independently.
\end{definition}

\begin{remark}[Tensor is Not Cartesian Product]
Unlike the cartesian product $A \times B$ which pairs elements, tensor $A \otimes B$ represents \emph{simultaneous independent interaction}. The positions of $A$ and $B$ coexist without interfering, but both must be satisfied.

This is the protocol-theoretic primitive---cartesian product will be a \emph{derived} construction (Section \ref{sec:ch1:classical_embedding}).
\end{remark}

\begin{theorem}[Tensor Properties]\label{thm:ch1:tensor_properties}
The tensor product $\otimes$ on $\Prot$ satisfies:
\begin{enumerate}[(a)]
\item \textbf{(Bifunctoriality)} For refinements $\sigma: P \to P'$ and $\tau: Q \to Q'$, there exists $\sigma \otimes \tau: P \otimes Q \to P' \otimes Q'$
\item \textbf{(Associativity)} $(P \otimes Q) \otimes R \cong P \otimes (Q \otimes R)$ (natural isomorphism)
\item \textbf{(Commutativity)} $P \otimes Q \cong Q \otimes P$ (braiding)
\item \textbf{(Unit)} $\mathbf{1} \otimes P \cong P \cong P \otimes \mathbf{1}$
\end{enumerate}
\end{theorem}

% PROOF_TODO: thm:ch1:tensor_properties

\subsection{Linear Implication and $*$-Autonomy}

\begin{definition}[Linear Implication]\label{def:ch1:linear_impl}
For protocols $P$ and $Q$, define the \emph{linear implication}:
\[
P \multimap Q := P^{\perp} \Par Q
\]
where $\Par$ is the \emph{par} connective (dual of tensor):
\[
A \Par B := (A^{\perp} \otimes B^{\perp})^{\perp}
\]
\end{definition}

\begin{remark}[Par is Simultaneous Demand]
While $A \otimes B$ means "provide both $A$ and $B$ simultaneously," $A \Par B$ means "demand both $A$ and $B$ simultaneously." These are De Morgan duals in the interaction sense.
\end{remark}

\begin{theorem}[Adjunction: Tensor and Linear Implication]\label{thm:ch1:tensor_linear_adj}
For any protocols $P, Q, R$, there is a natural bijection:
\[
\Prot(P \otimes Q, R) \cong \Prot(P, Q \multimap R)
\]
This makes $\otimes$ left adjoint to $\multimap$.
\end{theorem}

% PROOF_TODO: thm:ch1:tensor_linear_adj

\begin{theorem}[$*$-Autonomous Structure]\label{thm:ch1:star_autonomous}
The category $\Prot$ is $*$-autonomous:
\begin{enumerate}[(a)]
\item $(\Prot, \otimes, \mathbf{1})$ is symmetric monoidal
\item Every object $P$ has a dual $P^{\perp}$
\item There exists a dualizing object $\bot = \mathbf{1}^{\perp}$ such that $P^{\perp} \cong [P, \bot]$ where $[P, Q] = P \multimap Q$
\item The double dual embedding $P \to P^{\perp\perp}$ is a natural isomorphism
\end{enumerate}
\end{theorem}

% PROOF_TODO: thm:ch1:star_autonomous

\begin{corollary}[Protocol Theory Generalizes Linear Logic]
$\Prot$ is a \emph{model} of multiplicative linear logic (MLL). Every MLL proof corresponds to a protocol morphism, and cut-elimination corresponds to protocol composition simplification.
\end{corollary}

\begin{remark}[Why $*$-Autonomy Matters]
Set theory gives cartesian closed categories (products and function spaces). But protocols are \emph{more symmetric}: we have both $A \otimes B$ and $A \Par B$, both $A \multimap B$ and its dual. This extra structure---$*$-autonomy---is what makes protocol theory powerful enough to internalize duality.

No cartesian category can be $*$-autonomous unless it's a preorder (trivial). Thus protocol theory is fundamentally richer than set-theoretic foundations.
\end{remark}

\section{The Polarization Decomposition Theorem}

\subsection{Polarity Balance and Spectral Decomposition}

\begin{definition}[Polarity Path Balance]\label{def:ch1:path_balance}
For a path $\pi = m_0 \to m_1 \to \cdots \to m_k$ in arena $A$, define its \emph{polarity balance}:
\[
\beta(\pi) = \sum_{i=0}^k \lambda_A(m_i) \in \mathbb{Z}
\]
where we encode $+ \mapsto 1$ and $- \mapsto -1$.

For a position $m \in |A|$, define:
\begin{align*}
\beta^+(m) &= \sup\{\beta(\pi) : \pi \text{ is a path from } \iota_A \text{ to } m\} \\
\beta^-(m) &= \inf\{\beta(\pi) : \pi \text{ is a path from } \iota_A \text{ to } m\}
\end{align*}
\end{definition}

\begin{definition}[Polarity Classification]\label{def:ch1:polarity_class}
A position $m \in |A|$ is:
\begin{itemize}
\item \textbf{Positive-dominant} if $\beta^-(m) > 0$ (all paths net-positive)
\item \textbf{Negative-dominant} if $\beta^+(m) < 0$ (all paths net-negative)
\item \textbf{Balanced} if $\beta^-(m) \leq 0 \leq \beta^+(m)$ (exists zero-balance path)
\end{itemize}
\end{definition}

\begin{theorem}[Polarization Decomposition Theorem]\label{thm:ch1:polarization_decomp}
Every protocol arena $A$ admits a canonical decomposition:
\[
A \cong A^+ \otimes A^- \otimes A^0
\]
where:
\begin{enumerate}[(a)]
\item $A^+ = \{m \in |A| : \beta^-(m) > 0\}$ (positive-dominant positions)
\item $A^- = \{m \in |A| : \beta^+(m) < 0\}$ (negative-dominant positions)
\item $A^0 = \{m \in |A| : \beta^-(m) \leq 0 \leq \beta^+(m)\}$ (balanced positions)
\end{enumerate}

Moreover:
\begin{enumerate}[(i)]
\item The decomposition is \textbf{unique} up to protocol isomorphism
\item $A^0$ is \textbf{maximal}: any balanced sub-arena embeds into $A^0$
\item \textbf{Duality exchanges}: $(A^+)^{\perp} = A^-$ and $(A^0)^{\perp} = A^0$
\item \textbf{Dimension additivity}: $\dim(A) = \dim(A^+) + \dim(A^-) + \dim(A^0)$
\end{enumerate}
\end{theorem}

% PROOF_TODO: thm:ch1:polarization_decomp

\begin{proof}[Proof sketch]
The key observation is that positions in different polarity classes cannot enable each other: if $m \in A^+$ (net-positive) and $n \in A^-$ (net-negative), then $m \not<_A n$ by alternation (transitioning from positive-dominant to negative-dominant would require reversing polarity bias, impossible in a directed path).

Thus the three components are \emph{independent}---no enabling relations cross component boundaries---justifying the tensor decomposition. Uniqueness follows from the canonical definition via path balances. Duality flips $\beta \mapsto -\beta$, exchanging $A^+$ and $A^-$ while fixing $A^0$.
\end{proof}

\begin{corollary}[Three Fundamental Interpretations]\label{cor:ch1:three_interpretations}
The polarization decomposition has interpretations across domains:
\begin{enumerate}[(a)]
\item \textbf{Computation}: $A^+$ = values (data), $A^-$ = continuations (control), $A^0$ = thunks (suspension)
\item \textbf{Statistics}: $A^+$ = observations (likelihood), $A^-$ = queries (inference), $A^0$ = sufficient statistics (equilibrium)
\item \textbf{Physics}: $A^+$ = creation operators, $A^-$ = annihilation operators, $A^0$ = conserved quantities
\end{enumerate}
\end{corollary}

\begin{remark}[Polarization is Universal Structure]
This theorem shows that every protocol contains three canonical layers. Classical mathematics only sees $A^0$ (the balanced part)---it misses the asymmetry between $A^+$ and $A^-$. Protocol theory makes all three layers explicit.
\end{remark}

\subsection{Dimension and Complexity}

\begin{definition}[Protocol Dimension]\label{def:ch1:dimension}
The \emph{dimension} of an arena $A$ is:
\[
\dim(A) = \sup_{m \in |A|} \text{depth}_A(m)
\]
where $\text{depth}_A(m)$ is the length of the longest path from $\iota_A$ to $m$.

For a protocol $P = (A_P, \Sigma_P)$, define $\dim(P) = \sup_{s \in \Sigma_P} |s|$ (longest play in strategy).
\end{definition}

\begin{theorem}[Dimension and Tensor]\label{thm:ch1:dimension_tensor}
For protocol arenas $A, B$:
\begin{enumerate}[(a)]
\item \textbf{(Additivity)} $\dim(A \otimes B) = \dim(A) + \dim(B)$
\item \textbf{(Duality)} $\dim(A^{\perp}) = \dim(A)$
\item \textbf{(Bound)} $\dim(A \multimap B) \geq \max(\dim(A), \dim(B))$
\item \textbf{(Decomposition)} $\dim(A) = \dim(A^+) + \dim(A^-) + \dim(A^0)$
\end{enumerate}
\end{theorem}

% PROOF_TODO: thm:ch1:dimension_tensor

\begin{corollary}[Dimension and Computational Complexity]
\begin{enumerate}[(a)]
\item Protocols with $\dim(P) = O(\log n)$ admit polynomial-time strategies
\item Protocols with $\dim(P) = O(n)$ correspond to $\mathbf{PSPACE}$
\item Exponential-time problems have $\dim(P) = \Theta(2^n)$
\end{enumerate}
\end{corollary}

\begin{remark}
Dimension in protocol theory plays the role that cardinality plays in set theory: it's the fundamental measure of "size." But dimension is \emph{computational}---it measures interaction depth, not static size.
\end{remark}

\section{Trace Structure and Feedback}

\subsection{The Trace Operator}

\begin{definition}[Protocol Trace]\label{def:ch1:trace_op}
For a morphism $f: A \otimes X \to B \otimes X$ (protocol that inputs/outputs $X$ in addition to $A/B$), define the \emph{trace}:
\[
\Tr^A_B(f): A \to B
\]
as the protocol obtained by:
\begin{enumerate}[(i)]
\item Identifying output $X$ with input $X$ (feedback loop)
\item Hiding the $X$ positions (internal state)
\item Retaining only $A \to B$ interface
\end{enumerate}
\end{definition}

\begin{theorem}[Trace Properties]\label{thm:ch1:trace_properties}
The trace $\Tr^A_B: \Prot(A \otimes X, B \otimes X) \to \Prot(A, B)$ satisfies:
\begin{enumerate}[(a)]
\item \textbf{(Naturality)} $\Tr$ is natural in $A$ and $B$
\item \textbf{(Dinaturality)} $\Tr$ is dinatural in $X$
\item \textbf{(Yanking)} $\Tr^A_B(\gamma \otimes \id_X) = \gamma$ for $\gamma: A \to B$
\item \textbf{(Superposition)} $\Tr^{A \otimes C}_{B \otimes D}(f \otimes g) = \Tr^A_B(f) \otimes \Tr^C_D(g)$
\item \textbf{(Vanishing)} $\Tr^A_B(\Tr^{A'}_{B'}(h)) = \Tr^{A \otimes A'}_{B \otimes B'}(h')$ for appropriate $h'$
\end{enumerate}
\end{theorem}

% PROOF_TODO: thm:ch1:trace_properties

\begin{corollary}[$\Prot$ is Traced Monoidal]
$(\Prot, \otimes, \mathbf{1}, \Tr)$ is a traced symmetric monoidal category in the sense of Joyal-Street-Verity.
\end{corollary}

\begin{theorem}[Fixed Points via Trace]\label{thm:ch1:fixed_points}
Every morphism $f: A \to A$ has a least fixed point:
\[
\mu f = \Tr^{\mathbf{1}}_A(f): \mathbf{1} \to A
\]
characterized by: $\mu f = f \circ \mu f$ and for any $g$ with $g = f \circ g$, we have $g = \mu f \circ h$ for some $h$.

Moreover, fixed points are \textbf{constructive}: $\mu f$ is computed by iterating $f$ until stabilization (guaranteed by well-foundedness).
\end{theorem}

% PROOF_TODO: thm:ch1:fixed_points

\begin{remark}[Recursion is Primitive]
Unlike set theory (which requires special axioms for recursion) or type theory (which builds recursion into the syntax), protocol theory makes recursion \emph{primitive} via trace. Every recursive definition is a traced protocol.
\end{remark}

\subsection{Trace and Proof Nets}

\begin{theorem}[Geometry of Interaction Correspondence]\label{thm:ch1:goi_correspondence}
There is an isomorphism between:
\begin{enumerate}[(a)]
\item Traced protocols in $\Prot$
\item Girard's Geometry of Interaction (GoI) proof nets
\end{enumerate}
such that:
\begin{itemize}
\item Protocol composition corresponds to proof net cut-elimination
\item Trace corresponds to contracting tensor links (feedback wires)
\item Dimension corresponds to proof complexity (number of alternations)
\end{itemize}
\end{theorem}

% PROOF_TODO: thm:ch1:goi_correspondence

\begin{corollary}[Protocols Realize Linear Logic]
Every linear logic proof corresponds to a unique protocol (up to isomorphism), and vice versa. Cut-elimination in logic = protocol simplification.
\end{corollary}

\section{Embedding Classical Mathematics}\label{sec:ch1:classical_embedding}

\subsection{The Depolarization Functor}

\begin{definition}[Depolarization]\label{def:ch1:depolarization}
Define the \emph{depolarization functor} $\mathcal{D}: \Prot \to \Set$ by:
\begin{itemize}
\item On objects: $\mathcal{D}(P) = \{s \in \Sigma_P : s \text{ is maximal}\}$ (completed plays, forgetting polarity structure)
\item On morphisms: $\mathcal{D}(f: P \to Q) = $ the function $\mathcal{D}(P) \to \mathcal{D}(Q)$ induced by $f$
\end{itemize}
\end{definition}

\begin{theorem}[Depolarization is Symmetric Monoidal]\label{thm:ch1:depolarization_monoidal}
$\mathcal{D}: \Prot \to \Set$ is a symmetric monoidal functor:
\begin{enumerate}[(a)]
\item $\mathcal{D}(\mathbf{1}) = \{*\}$ (singleton)
\item $\mathcal{D}(P \otimes Q) \cong \mathcal{D}(P) \times \mathcal{D}(Q)$ (tensor becomes product)
\item $\mathcal{D}$ preserves all limits
\end{enumerate}

Moreover, $\mathcal{D}$ has a full and faithful left adjoint:
\[
\mathcal{S}: \Set \to \Prot
\]
sending sets to sampling protocols (Example \ref{ex:ch1:basic_arenas}(c)).
\end{theorem}

% PROOF_TODO: thm:ch1:depolarization_monoidal

\begin{corollary}[Set Theory Embeds into Protocol Theory]
$\Set$ embeds fully and faithfully into $\Prot$ via $\mathcal{S}$. Every set is a protocol (sampling), every function is a protocol morphism.
\end{corollary}

\begin{theorem}[What Depolarization Forgets]\label{thm:ch1:depolarization_forgets}
The functor $\mathcal{D}$ is \textbf{not} full or faithful. Specifically:
\begin{enumerate}[(a)]
\item \textbf{(Non-injectivity)} Distinct protocols can have identical depolarizations: call-by-value and call-by-name both depolarize to the same function space
\item \textbf{(Non-surjectivity)} Not every function $\mathcal{D}(P) \to \mathcal{D}(Q)$ lifts to a protocol morphism: some functions violate causality (output before input)
\item \textbf{(Complexity loss)} Protocols with vastly different computational complexity can have identical depolarizations
\end{enumerate}
\end{theorem}

% PROOF_TODO: thm:ch1:depolarization_forgets

\begin{remark}[Classical Mathematics = Depolarized Protocols]
Set theory is the result of \emph{forgetting polarity}. This explains why classical mathematics cannot naturally express:
\begin{itemize}
\item Evaluation order (call-by-value vs. call-by-name)
\item Resource usage (time, space, parallelism)
\item Causality (what depends on what)
\end{itemize}
These are inherently polarized phenomena that depolarization erases.
\end{remark}

\subsection{Cartesian Product as Derived Construction}

\begin{definition}[Cartesian Product of Protocols]\label{def:ch1:cartesian_product}
For protocols $P, Q$, define their \emph{cartesian product}:
\[
P \times Q = \mathcal{D}^{-1}(\mathcal{D}(P) \times \mathcal{D}(Q))
\]
where $\mathcal{D}^{-1}$ is the left adjoint $\mathcal{S}$ applied to the set-theoretic product.

Explicitly: $P \times Q$ is the protocol that samples from $P$ and $Q$ \emph{sequentially} (not simultaneously like $\otimes$), then pairs results.
\end{definition}

\begin{theorem}[Tensor vs. Cartesian Product]\label{thm:ch1:tensor_vs_product}
The tensor $P \otimes Q$ and cartesian product $P \times Q$ are \textbf{distinct} in general:
\begin{enumerate}[(a)]
\item $P \otimes Q$ represents simultaneous independent interaction
\item $P \times Q$ represents sequential interaction with result pairing
\item $\mathcal{D}(P \otimes Q) = \mathcal{D}(P \times Q)$ (indistinguishable in set theory)
\item But $\dim(P \otimes Q) = \dim(P) + \dim(Q)$ while $\dim(P \times Q) = \max(\dim(P), \dim(Q))$
\end{enumerate}
\end{theorem}

% PROOF_TODO: thm:ch1:tensor_vs_product

\begin{corollary}[Cartesian Closed Structure]
$\Prot$ is cartesian closed with respect to $\times$:
\[
\Prot(P \times Q, R) \cong \Prot(P, Q \Rightarrow R)
\]
where $Q \Rightarrow R$ is the function space (Example \ref{ex:ch1:basic_arenas}(d)).

But $\Prot$ is \emph{also} $*$-autonomous with respect to $\otimes$. This dual structure---both cartesian and $*$-autonomous---is what makes protocol theory expressive.
\end{corollary}

\section{The Master Embedding Theorem}

\subsection{Universal Property of Protocol Theory}

\begin{theorem}[Master Embedding Theorem]\label{thm:ch1:master_embedding}
For any category $\mathcal{C}$ satisfying:
\begin{enumerate}[(i)]
\item $\mathcal{C}$ has finite products and coproducts
\item $\mathcal{C}$ is cartesian closed
\item $\mathcal{C}$ has a natural numbers object (NNO)
\end{enumerate}
there exists a full and faithful functor:
\[
\Phi_{\mathcal{C}}: \mathcal{C} \to \Prot
\]
preserving all limits, colimits, and exponentials.

Moreover, if $\mathcal{C}$ has additional structure:
\begin{enumerate}[(a)]
\item \textbf{(Topology)} If $\mathcal{C} = \Top$, then $\Phi$ preserves continuous maps and convergence
\item \textbf{(Measure)} If $\mathcal{C} = \Meas$, then $\Phi$ preserves measurable functions and integration
\item \textbf{(Algebra)} If $\mathcal{C} = \mathbb{T}\text{-}\cat{Alg}$ for algebraic theory $\mathbb{T}$, then $\Phi$ preserves operations and equations
\end{enumerate}
\end{theorem}

% PROOF_TODO: thm:ch1:master_embedding

\begin{proof}[Proof sketch]
The key is defining $\Phi$ on objects via "interaction protocols for $\mathcal{C}$-objects":
\begin{itemize}
\item For $X \in \mathcal{C}$, define $\Phi(X)$ as the protocol: "demand a $\mathcal{C}$-morphism $\mathbf{1} \to X$ (global element), provide the result"
\item Morphisms $f: X \to Y$ become protocol refinements: "given input protocol for $X$, compose with $f$, produce output protocol for $Y$"
\item Limits/colimits in $\mathcal{C}$ become limits/colimits of interaction protocols
\end{itemize}

Faithfulness: distinct morphisms in $\mathcal{C}$ induce distinct protocol transformations.
Fullness: every protocol morphism $\Phi(X) \to \Phi(Y)$ arises from some $f: X \to Y$ by universality.

Additional structure preservation follows from enriching the protocol with appropriate observables (opens for topology, measurable sets for measure theory, operations for algebra).
\end{proof}

\begin{corollary}[Everything is a Protocol]
Every mathematical structure---sets, topological spaces, measure spaces, groups, rings, vector spaces---has a canonical protocol representation. Mathematics is the study of protocols.
\end{corollary}

\subsection{Uniqueness of Protocol Theory}

\begin{theorem}[Universal Characterization]\label{thm:ch1:universal_char}
The category $\Prot$ is characterized uniquely (up to equivalence) as:

\textbf{The initial $*$-autonomous traced symmetric monoidal category with:}
\begin{enumerate}[(i)]
\item An involutive duality $(-)^{\perp}$ satisfying $(A^{\perp})^{\perp} \cong A$
\item A trace $\Tr$ satisfying Joyal-Street-Verity axioms
\item A dimension function $\dim: \Prot \to \mathbb{N} \cup \{\omega\}$ satisfying additivity under $\otimes$
\item A depolarization functor $\mathcal{D}: \Prot \to \Set$ with full and faithful left adjoint
\end{enumerate}

That is: any other category with these properties admits a unique (up to isomorphism) structure-preserving functor from $\Prot$.
\end{theorem}

% PROOF_TODO: thm:ch1:universal_char

\begin{corollary}[Protocol Theory is Inevitable]
Protocol theory is not an arbitrary choice of foundation---it is the \textbf{unique} foundation satisfying natural requirements for interactive computation. You cannot choose the axioms differently and get something else; the structure is mathematically determined.
\end{corollary}

\section{Coherence and Equality}

\subsection{Protocol Coherence Theorem}

\begin{definition}[Canonical Isomorphisms]\label{def:ch1:canonical_isos}
The \emph{canonical isomorphisms} in $\Prot$ are:
\begin{enumerate}[(i)]
\item Associators: $\alpha_{A,B,C}: (A \otimes B) \otimes C \to A \otimes (B \otimes C)$
\item Unitors: $\lambda_A: \mathbf{1} \otimes A \to A$ and $\rho_A: A \otimes \mathbf{1} \to A$
\item Symmetry: $\sigma_{A,B}: A \otimes B \to B \otimes A$
\item Duality pairings: $\eta_A: \mathbf{1} \to A \otimes A^{\perp}$ and $\epsilon_A: A^{\perp} \otimes A \to \mathbf{1}$
\end{enumerate}
\end{definition}

\begin{theorem}[Coherence for $\Prot$]\label{thm:ch1:coherence}
In $\Prot$, every diagram built from canonical isomorphisms commutes. Specifically:
\begin{enumerate}[(a)]
\item \textbf{(Mac Lane)} All compositions of associators, unitors, and symmetries commute
\item \textbf{(Kelly-Laplaza)} The duality pairings satisfy yanking identities
\item \textbf{(Trace)} All trace naturality and dinaturality diagrams commute
\end{enumerate}

Consequently, protocol morphisms are unique up to canonical isomorphism: if $f, g: P \to Q$ differ only by canonical isomorphisms, then $f = g$ in the quotient by canonical isomorphisms.
\end{theorem}

% PROOF_TODO: thm:ch1:coherence

\begin{corollary}[Equality is Decidable]
Equality of protocol morphisms (up to canonical isomorphisms) is decidable: reduce both sides to normal form using coherence, then compare syntactically.
\end{corollary}

\subsection{Rewriting and Normal Forms}

\begin{definition}[Protocol Rewrite System]\label{def:ch1:rewrite_system}
Define rewrite rules on protocols:
\begin{enumerate}[(R1)]
\item $\Tr^A_B(\gamma \otimes \id) \to \gamma$ (yanking)
\item $\Tr^A_B(\Tr^{A'}_{B'}(f)) \to \Tr^{A \otimes A'}_{B \otimes B'}(f')$ (trace fusion)
\item $(f \otimes g) \circ \alpha \to \alpha \circ (f \otimes g)$ (naturality)
\item $f \circ \id \to f \to \id \circ f$ (identity elimination)
\end{enumerate}
\end{definition}

\begin{theorem}[Confluence and Termination]\label{thm:ch1:confluence_termination}
The protocol rewrite system is:
\begin{enumerate}[(a)]
\item \textbf{(Terminating)} Every rewrite sequence eventually reaches a normal form
\item \textbf{(Confluent)} If $P \to^* Q$ and $P \to^* R$, then exists $S$ with $Q \to^* S$ and $R \to^* S$ (Church-Rosser)
\item \textbf{(Unique normal forms)} Normal forms are unique up to $\alpha$-equivalence
\end{enumerate}

Moreover, normal forms correspond to \textbf{geodesics} in protocol space: protocols in normal form have minimal complexity for their behavior.
\end{theorem}

% PROOF_TODO: thm:ch1:confluence_termination

\begin{corollary}[Protocol Verification is Decidable]
To verify $P \cong Q$ (protocol equivalence):
\begin{enumerate}[(i)]
\item Reduce $P$ and $Q$ to normal forms $P'$ and $Q'$
\item Check syntactic equality $P' = Q'$
\end{enumerate}
This is decidable for finite protocols and semi-decidable (c.e.) for infinite protocols.
\end{corollary}

\section{Applications and Interpretations}

\subsection{Computation as Protocols}

\begin{theorem}[Lambda Calculus Embedding]\label{thm:ch1:lambda_embedding}
The simply-typed lambda calculus embeds fully and faithfully into $\Prot$ such that:
\begin{enumerate}[(a)]
\item Types $\tau$ become protocols $\llbracket \tau \rrbracket$
\item Terms $\Gamma \vdash M: \tau$ become morphisms $\llbracket \Gamma \rrbracket \to \llbracket \tau \rrbracket$
\item $\beta$-reduction corresponds to protocol refinement
\item $\eta$-equality corresponds to trace yanking
\end{enumerate}

Moreover, operational semantics (call-by-value, call-by-name, call-by-need) distinguish protocols that depolarize to the same set-theoretic function.
\end{theorem}

% PROOF_TODO: thm:ch1:lambda_embedding

\subsection{Game Semantics as Protocol Fragments}

\begin{theorem}[Game Semantics Correspondence]\label{thm:ch1:game_semantics}
Abramsky-Jagadeesan-Malacaria game semantics embeds into $\Prot$ via:
\begin{itemize}
\item Game arenas $\leadsto$ Protocol arenas
\item Player strategies $\leadsto$ Positive-polarity protocols ($A^+$)
\item Opponent strategies $\leadsto$ Negative-polarity protocols ($A^-$)
\item Game composition $\leadsto$ Protocol tensor with hiding
\end{itemize}

The embedding is full and faithful, preserving:
\begin{enumerate}[(a)]
\item Innocent strategies $\leftrightarrow$ Finite-dimension protocols
\item Winning strategies $\leftrightarrow$ Total protocols (all negative positions have responses)
\item Copycat strategies $\leftrightarrow$ Identity protocols
\end{enumerate}
\end{theorem}

% PROOF_TODO: thm:ch1:game_semantics

\begin{corollary}[Game-Theoretic Foundations Enhance Protocols]
Protocol theory provides the computational substrate for game-theoretic foundations, answering:
\begin{itemize}
\item \emph{What strategies exist?} (game theory)
\item \emph{How do we compute them?} (protocol theory)
\end{itemize}
\end{corollary}

\subsection{Optimal Transport as Protocol Metrics}

\begin{theorem}[Wasserstein Protocols]\label{thm:ch1:wasserstein_protocols}
For stochastic protocols $P_\mu, P_\nu$ (protocols with probability measures on traces), the Wasserstein distance:
\[
W_p(\mu, \nu) = \inf_{\pi \in \Pi(\mu, \nu)} \mathbb{E}_{(x,y) \sim \pi}[d(x,y)^p]^{1/p}
\]
equals the protocol refinement cost:
\[
W_p(\mu, \nu) = \inf_{\rho: P_\mu \Rightarrow P_\nu} \cost(\rho)
\]
where cost is the expected trace transformation expense.
\end{theorem}

% PROOF_TODO: thm:ch1:wasserstein_protocols

\begin{corollary}[OT Foundations Enhance Protocols]
Optimal transport provides the metric structure for protocols, answering:
\begin{itemize}
\item \emph{What is the optimal coupling?} (OT theory)
\item \emph{How do we sample from it?} (protocol theory)
\end{itemize}
\end{corollary}

\section{Conclusion and Forward References}

\subsection{What We Have Established}

In this chapter, we have:
\begin{enumerate}
\item Defined protocols from first principles, with polarity as the primitive asymmetry
\item Proved protocols form a $*$-autonomous traced monoidal category
\item Established the Polarization Decomposition Theorem: every protocol factors into $A^+ \otimes A^- \otimes A^0$
\item Shown classical mathematics embeds via depolarization, explaining why set theory misses computational structure
\item Proved protocol theory is uniquely characterized by universal properties
\item Demonstrated embeddings from lambda calculus, game semantics, and optimal transport
\end{enumerate}

\subsection{What Comes Next}

Subsequent chapters will develop:
\begin{itemize}
\item \textbf{Chapter 2:} Higher-dimensional protocols and homotopy structure
\item \textbf{Chapter 3:} Stochastic protocols and probabilistic verification
\item \textbf{Chapter 4:} Distributional protocols and Wasserstein geometry
\item \textbf{Chapter 5:} Protocol logic and modal reasoning
\item \textbf{Chapter 6:} Algorithm design via protocol geometry
\item \textbf{Chapter 7:} Protocol cohomology and obstruction theory
\item \textbf{Chapter 8:} Computational complexity via dimension
\end{itemize}

\subsection{The Central Message}

\begin{mdframed}[linecolor=black, linewidth=1pt]
\textbf{Protocol theory is not "just another foundation."}

It is the unique mathematical framework that:
\begin{itemize}
\item Makes interaction primitive (not derived)
\item Internalizes duality ($*$-autonomy)
\item Supports recursion natively (trace)
\item Embeds all classical mathematics (depolarization)
\item Provides computational content (dimension, complexity)
\item Unifies logic, computation, and geometry
\end{itemize}

Every other foundation is either:
\begin{enumerate}[(a)]
\item \textbf{Less expressive}: Set theory cannot express polarity; category theory cannot internalize duality
\item \textbf{Complementary}: Game theory provides strategic structure; OT provides metric structure; protocols provide computational structure
\item \textbf{Derivable}: Classical mathematics = depolarized protocols
\end{enumerate}

\textbf{Protocols are not "models of" mathematics. Mathematics is the study of protocols.}
\end{mdframed}

\begin{remark}[Philosophical Shift]
The protocol-theoretic view transforms foundational questions:
\begin{center}
\begin{tabular}{l|l}
\textbf{Classical question} & \textbf{Protocol question} \\ \hline
"Does $X$ exist?" & "What protocol computes $X$?" \\
"Is $f(x) = y$?" & "Does protocol $f$ refine specification $(x,y)$?" \\
"Are $A$ and $B$ isomorphic?" & "What is $\cost(A \Rightarrow B)$ and $\cost(B \Rightarrow A)$?" \\
"Is the proof correct?" & "Does the refinement verify?"
\end{tabular}
\end{center}

Mathematics becomes inherently constructive, computational, and verifiable.
\end{remark}

\newpage

\section*{Summary of Main Theorems}

For quick reference, the major theorems of this chapter:

\begin{enumerate}
\item \textbf{Theorem \ref{thm:ch1:prot_category}}: Protocols form a category $\Prot$
\item \textbf{Theorem \ref{thm:ch1:involutive_duality}}: Duality is involutive: $(A^{\perp})^{\perp} = A$
\item \textbf{Theorem \ref{thm:ch1:strategy_env_dual}}: Strategy-environment duality bijection
\item \textbf{Theorem \ref{thm:ch1:tensor_properties}}: Tensor product properties (associative, commutative, unital)
\item \textbf{Theorem \ref{thm:ch1:tensor_linear_adj}}: Tensor-linear implication adjunction
\item \textbf{Theorem \ref{thm:ch1:star_autonomous}}: $\Prot$ is $*$-autonomous
\item \textbf{Theorem \ref{thm:ch1:polarization_decomp}}: Polarization Decomposition $A \cong A^+ \otimes A^- \otimes A^0$
\item \textbf{Theorem \ref{thm:ch1:dimension_tensor}}: Dimension and tensor additivity
\item \textbf{Theorem \ref{thm:ch1:trace_properties}}: Trace operator properties (naturality, yanking, superposition)
\item \textbf{Theorem \ref{thm:ch1:fixed_points}}: Constructive fixed points via trace
\item \textbf{Theorem \ref{thm:ch1:goi_correspondence}}: Geometry of Interaction correspondence
\item \textbf{Theorem \ref{thm:ch1:depolarization_monoidal}}: Depolarization is symmetric monoidal
\item \textbf{Theorem \ref{thm:ch1:depolarization_forgets}}: What depolarization forgets (computational structure)
\item \textbf{Theorem \ref{thm:ch1:tensor_vs_product}}: Tensor vs cartesian product distinction
\item \textbf{Theorem \ref{thm:ch1:master_embedding}}: Master Embedding: every category embeds into $\Prot$
\item \textbf{Theorem \ref{thm:ch1:universal_char}}: Universal characterization of protocol theory
\item \textbf{Theorem \ref{thm:ch1:coherence}}: Coherence for canonical isomorphisms
\item \textbf{Theorem \ref{thm:ch1:confluence_termination}}: Rewrite system confluence and termination
\item \textbf{Theorem \ref{thm:ch1:lambda_embedding}}: Lambda calculus embedding
\item \textbf{Theorem \ref{thm:ch1:game_semantics}}: Game semantics correspondence
\item \textbf{Theorem \ref{thm:ch1:wasserstein_protocols}}: Wasserstein distance = protocol refinement cost
\end{enumerate}

All proofs will be formalized in Lean 4 in the accompanying development.

\end{document}
